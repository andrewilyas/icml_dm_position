\documentclass{article}
\usepackage{amsmath}
\usepackage{amssymb}
\usepackage{amsthm}
\usepackage{graphicx}
\usepackage{hyperref}
\usepackage{geometry}
\usepackage{xcolor}
\usepackage{todonotes}

\usepackage[natbib=true]{biblatex}
\addbibresource{ref.bib}
\geometry{a4paper, margin=1in}

\title{Position: Interpretability Should be Data-First}
\author{Chungus P. Fungus}

\begin{document}
\maketitle
\begin{abstract}
    We propose that interpretability should be data-first.
    Even when the final goal is bla bla bla.
\end{abstract}

\section{Introduction}

\section{Background: Interpretability}
The goal of interpretability is to understand ``why'' a
machine learning model arrived at a given prediction.
We might need this understanding
to comply with regulation,
ensure fairness, 
or fix incorrect predictions.
In this work, we'll roughly categorize interpretability research 
into three guiding questions:
\begin{enumerate}
    \item {\em Would a \underline{\smash{specific}} change to my input change the prediction?}
    This kind of question appears in contexts like that of fairness and robustness, 
    where we often want to make sure that changing, e.g., the race of
    an applicant \citep{TODO} or only the background of an image 
    \citep{xiao2021noise} will {\em not} change a machine learning model's prediction.
    \item {\em What change to the input \underline{would} change the prediction?} 
    This type of question is also known as {\em recourse},
    and is an important way to both assess the robustness of machine learning models
    and also give decision subjects an avenue to improve their predictions.
    \item {\em What aspects of the input bear responsibility for this prediction?}
    This is perhaps the least well-formed of the three questions, 
    but also the most aligned with what we might colloquially call ``interpretability.''
    This question, for example, is what methods like feature attribution 
    (e.g., SHAP, LIME, GradCAM) and visualization methods aim to tackle.
\end{enumerate}
All three of these questions have proved extremely difficult for a 
variety of different reasons.
The third question, which we'll call ``transparency,'' 
is the most difficult while also being the one most commonly discussed.
Many prior works have challenged legitimacy and reliability of existing
interpretability methods.
Phenomena such as model-invariance \citep{sanitychecks},
superposition \citep{superposition},
and adversarial examples \citep{adversarialexamples}
all pose their own respective threats.

\section{Feature-space vs data-space interpretability}
In this section, we overview the two main approaches to interpretability
that we will discuss in this paper.

\subsection{Feature-space interpretability}
\todo{Explain feature-space interpretability}

\subsection{Data-first interpretability}
Broadly speaking, data-first interpretability is the idea that we should
explain model behavior by looking at the model's training data.
More concretely: so far we have thought of the ``input'' as a single example
(for example, an image or a sentence), and the ``model'' as a function---defined 
by a set of parameters---that takes an input and returns a prediction.
The data-first perspective instead thinks of the ``input'' as 
a dataset, and the ``model'' as a function---defined by a {\em learning
algorithm}---that takes a dataset and returns a prediction.

\paragraph{Data attribution.} Much of the fuel behind this idea comes from
recent advances in {\em training data attribution} \citep{}, allowing 
\todo{talk about datamodels}

\subsection{Out of scope: mechanistic interpretability}
Another emergent field of interpretability research is that of mechanistic interpretability,
which seeks to understand the inner workings of a model by decomposing it into its 
constituent parts.
To keep this paper focused, we will not discuss mechanistic interpretability 
at length.
In general, while the jury is still out on whether this approach can successfully 
explain model behavior, we believe that it must contend with 
(and come up with solutions to)
all the challenges to feature-space interpretability 
that we discuss in this work.


\section{The challenge of interpretability}
\todo{Integrate this section into the next one - the idea being that
true interpretability requires a structural causal model, 
and that model is easier to specify and test in data space.}
Why is interpretability so hard? 
We argue that the key to answering this question 
is the (already well-known) realization that at its core,
\begin{center}
{\em interpretability is a causal inference problem} \citep{chungus,fungus,mungus}.
\end{center}
After all, a ``change'' to an input is simply an intervention or treatment 
applied to that input, and a ``prediction'' is simply a measurable outcome. 
And causal inference is all about understanding the relationship between
treatments and outcomes.

A causal perspective on interpretability allows us to understand the 
desiderata associated with each of the three questions above.
The first question simply asks us to measure the effect of a 
pre-specified treatment on an outcome.
The second question asks us to {\em find} a treatment that would yield a 
desired outcome.
The third question, on the other hand, is only answerable with a 
complete causal model of the function in question.



\section{Three advantages of data-first interpretability}
Data-based transparency is not intrinsically ``better'' than its feature-based counterpart. 
Both types of transparency seek to explain complex functions from 
a high-dimensional input space to a prediction---for data-based explanations, 
the input space is the space of training datasets and the complex function is 
the model output function, while for feature-based explanations, the input space 
is the space of input examples and the complex function is a forward pass through 
the model.
Rather than being inherently better, the advantages of data-first interpretability 
come from the structure of this input space, and its relationship to the model's 
behavior. In this position paper, we focus on three such advantages, 
which we describe below.

\subsection{Removal is well-defined}
A common motif in feature-based interpretability is that of ``removing'' a
feature from an input.
For example, LIME \citep{ribeiro2016should}

This is simple for humans, but far more challenging for machine learning models
(Sturmfels et al., 2020). 
We cannot leave a “hole" in the image where 
the model should ignore the input; similarly, it is difficult to 
``remove'' the style from a sentence.

When the input is a dataset, however, the concept of ``removing'' a feature is 
well-defined---we can simply remove an example (or set of examples) from the 
training dataset.



\subsection{The data landscape is more well-behaved}

\subsection{Data counterfactuals are more actionable}
Finally, we argue that data-first interpretability is more actionable 
for the model developer, making it more suited to applications where 
the end goal is to understand and debug the model's behavior.

\section{Open directions}

\subsection{Lifting to feature space}
\subsection{Curbing misinterpretation}

\paragraph{Improving usability.}
Despite the existence of several performant data-based interpretability methods
in the form of data attribution algorithms, practitioner uptake of these 
tools has been slow. 
Recent studies \citep{nguyen2023exploring} suggest 
that part of the reason for this is that these tools lack 
usability.
For example, ...
Future work on data-first interpretability should thus focus not only 
on improving the performance of these tools, 
but also on conveying their results to end users in a meaningful way.

\subsection{Global verification and uncertainty}

\section{Limitations of data-based interpretability}

\paragraph{Non-determinism.}
Almost by construction, we cannot use data-based interpretability to 
understand the behavior of a specific model, but rather only the behavior 
of the family of models induced by a learning algorithm.
After all, if training on the same dataset yields different models, 
the differences in these models' behavior cannot be attributed to 
training data.

\paragraph{Potential for misinterpretation.} It is also important to
note that even when we have a complete causal model of a function,
the way we present this model can lead to misinterpretation.

\section{Alternative views}
\todo{As specified by the CFP}

\section{Related Work}

\section{Conclusion}

\printbibliography

\end{document}